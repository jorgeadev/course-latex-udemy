\documentclass[12pt,letterpaper]{article}

\usepackage[utf8]{inputenc}
\usepackage[spanish]{babel}
\usepackage[T1]{fontenc}
\usepackage{lmodern}

\usepackage{amsmath}
\usepackage{amsfonts}
\usepackage{amssymb}

\usepackage{graphicx}
\usepackage[colorlinks = true, linkcolor = blue]{hyperref}
\usepackage{fancyhdr}
\usepackage{enumerate}
\usepackage{tabu}
\usepackage{float}
\usepackage{enumitem}

\newcommand{\sen}{$\sen$}
\spanishdecimal{.}

\usepackage[none]{hyphenat}
\setlength{\parindent}{0 pt}

\usepackage{kpfonts}
\usepackage{makeidx}
\usepackage{xcolor}
\usepackage[left = 2cm, right = 2cm, top = 2cm, bottom = 2cm]{geometry}

\author{Jorge Alberto Gomez Gomez}
\title{Sistemas eléctricos. Monofásico. Bifásico. Trifásico}
\date{}

\begin{document}
% pie de página
\pagestyle{fancy}
\thispagestyle{empty}
\fancyhf{}
\lfoot[]{}
\cfoot[]{}
\rfoot[]{Jorge Alberto Gomez Gomez}
\rhead[]{t\hepage}
\renewcommand{\headrulewidth}{0 pt}
\renewcommand{\footrulewidth}{0 pt}

\sloppy
\maketitle

\tableofcontents

\newpage

\section{Definición de la NOM-001-SEDE-2012}

\begin{tabu}{|l|l|}
\hline
1 & a \\ \hline
2 & b \\ \hline
3 & c \\ \hline
\end{tabu}


Las siguientes definiciones sirven para comprender y explicar los diferentes sistemas eléctricos:

\begin{enumerate}

\item \textit{\textbf{Tensión (de un circuito):} La mayor diferencia de potencial (tensión rms) entre dos conductores cualesquiera de un circuito considerado.}

\item \textit{\textbf{Tensión a tierra:} En los circuitos puestos a tierra, es la tensión entre un conductor dado y el punto o conductor del circuito que está puesto a tierra; en circuitos no puestos a tierra es la mayor diferencia de potencial entre un conductor dado y cualquier otro conductor del circuito.}

\textit{\textbf{NOTA:} Algunos sistemas, como como los de 3 fases 4 hilos, de 1 fase 3 hilos y de corriente continua de 3 hilos, pueden tener varios circuitos a diferentes tensiones.}

\item \textit{\textbf{Tensión nominal:} Valor nominal asignado a un circuito o sistema para designar convenientemente su clase de tensión. La tensión a la cual un circuito opera puede variar de la nominal, dentro de un margen que permite el funcionamiento satisfactorio de los equipos.}

\textit{\textbf{NOTA:} Donde se lea 120 volts, podrá ser 120 o 127 volts.\\}

\item \textit{\textbf{Conductor neutro:} Conductores conectados al punto neutro de un sistema que está destinado a transportar corriente en condiciones normales.\\}

\item \textit{\textbf{Puesto a tierra:} Conectado (conexión) o a algún cuerpo conductor que extienda la conexión a tierra.}

\item \textit{\textbf{Puesto a tierra eficazmente:} Conectado (conexión) a tierra intencionalmente a }

\end{enumerate}

\end{document}





















